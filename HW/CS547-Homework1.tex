\documentclass{article}
	\usepackage{listings}
	\author{Peng Li}
	\title{CS547-Homework1}
	\begin{document}
		\maketitle
		\paragraph{1} For example: we search some news about football match, this is not ad-hoc retrieval. The problem is that may be new documents just come in when we already got our retrieval result and we don’t know is the new document is better.
		\paragraph{2} For example, there is a news from web page \\
		\begin{lstlisting}[language=html]
<article id="article" class="article">
<header><div class="byline">
<span class="source">AP</span>
<span class="time">January 24, 2016, 10:04 AM</span></div>
<h1 class="title" itemprop="headline">Flight 370 speculation 
surrounds Thai beach debris</h1>
</header><div class="article-image" itemprop="image">
<p><strong>BANGKOK </strong>- A large chunk 
of metal that could be from an aircraft washed ashore in southern 
Thailand, but Malaysian authorities on Sunday cautioned against 
speculation of a link to a <a href="http://www.cbsnews.com/
malaysia-airlines-flight-370/" target="_blank">Malaysia Airlines 
Flight 370</a> missing almost two years.</p><p>Flight 370 lost 
communications and made a sharp turn away from its Beijing 
destination before disappearing in March 2014. It is presumed 
to have crashed in the Indian Ocean, and only one piece of debris 
has been identified as coming from the plane, a slab of wing that
 washed ashore on Reunion Island in the western Indian Ocean last 
 July.</p><p>Malaysian Transport Minister Liow Tiong Lai said he 
 instructed Malaysian civil aviation officials to contact Thailand 
 about the newly found wreckage, a curved piece of metal measuring 
 6.5 feet by 10 feet with electrical wires hanging from it and  
 numbers stamped on it in several places.</p><p>"I urge the media 
 and the public not to speculate because it will give undue pressure 
 to the loved ones of the victims of flight 370," he said.</p>
 <p>Thailand's Transportation Ministry said four Malaysian officials 
 and two Thai experts will visit the site Monday.</p></div>
</article>
		\end{lstlisting}
		
	I focus on MH370 and my query is "MH370 aeroplane crash". Only number "370" appears  in this article. So, if we use all word in this html file when we preprocess, we will not get this article or this article will have a low rank among all result although this article has a high relevance with my searching key word. If this article is preprocessed based on structural properties, I will get better result.

	\paragraph{3.1 controlled vocabulary - recall.} Indexing text with controlled vocabulary is easy to cross vocabulary gap which means user will get more relevant documents and it will improve recall, but maybe get more irrelevent documents. For example: my query is “vehicle” and I can get many documents with key word “car” and “auto”, but also can get excipient.
	\paragraph{3.2 remove stopwords - both} When I search “Purdue”,I will not get a lot of result about “the” or “is” although this word appear more times than other words because stopwords are removed. So I think stopwords removing will improve both precision and recall.
	\paragraph{3.3 stemming - recall} Stemming can match word better but it also have disadvantage. When I search “American” and I want to know more about America people, may be I will get lots of results about “America”. The precision will be decreased but I think I will get more relevant document and it can improve recall. 
	\paragraph{3.4 Representing phrases as a singer term - precision} When I query “hit a home run”, I will get many results related to baseball because “hit a home run” in there is a baseball term rather than other things, such as "home","hit","run".
	\paragraph{4}
	In most case, we get many results means these results have a low precision and a high recall. This is because the more results we get, the more relevant results we get. So we can get a high recall. But we get more wrong result at the same time. So we get a low precision. If we get less results and they are relevant, we can get a high precision. Although we have a high precision, we have a low recall because we get only little part of all relevant results.  So, there is a trade off between precision and recall. In that case, Balanced F-measure will not have much change for different information retrieval system. It can't correctly evaluation an information retrieval system. That’s why balanced F-measure is not good for information retrieval system evaluation. 
	\paragraph{5.1}
	\paragraph{}
	\begin{tabular}{|c|c|c|}
		\hline
		Term ID & Term & Documents\\
		\hline
		1 & Purdue & 1,3,4 \\
		\hline
		2 & Information & 1,4 \\
		\hline
		3 & Retrieval & 1,2,3,4 \\
		\hline
	\end{tabular}
	\paragraph{5.2}
	\paragraph{}
	\begin{flushleft}
	"Purdue Information" : $\vec{q_1}$ = (1,1,0) \\
	 Doc3 : $\vec{q_2}$ = (2,0,3) \\
	$cos(\vec{q_1},\vec{q_2}) = \frac{1 \times 2 + 1 \times 0 + 0 \times 3}{\sqrt{1^2 + 1^2 + 0^2}  \sqrt{2^2 + 0^2 + 3^2}} = \frac{\sqrt{26}}{13} $  \\ 
	when I am search "Purdue Information", I think Doc1 will be ranked higher
	\paragraph{5.3}
	\paragraph{} I think Doc4 is most likely about our class "Information Retrieval"
	\end{flushleft} 
	\end{document}